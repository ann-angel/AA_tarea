% Options for packages loaded elsewhere
\PassOptionsToPackage{unicode}{hyperref}
\PassOptionsToPackage{hyphens}{url}
\PassOptionsToPackage{dvipsnames,svgnames,x11names}{xcolor}
%
\documentclass[
  super,
  preprint,
  3p]{elsarticle}

\usepackage{amsmath,amssymb}
\usepackage{iftex}
\ifPDFTeX
  \usepackage[T1]{fontenc}
  \usepackage[utf8]{inputenc}
  \usepackage{textcomp} % provide euro and other symbols
\else % if luatex or xetex
  \usepackage{unicode-math}
  \defaultfontfeatures{Scale=MatchLowercase}
  \defaultfontfeatures[\rmfamily]{Ligatures=TeX,Scale=1}
\fi
\usepackage{lmodern}
\ifPDFTeX\else  
    % xetex/luatex font selection
\fi
% Use upquote if available, for straight quotes in verbatim environments
\IfFileExists{upquote.sty}{\usepackage{upquote}}{}
\IfFileExists{microtype.sty}{% use microtype if available
  \usepackage[]{microtype}
  \UseMicrotypeSet[protrusion]{basicmath} % disable protrusion for tt fonts
}{}
\makeatletter
\@ifundefined{KOMAClassName}{% if non-KOMA class
  \IfFileExists{parskip.sty}{%
    \usepackage{parskip}
  }{% else
    \setlength{\parindent}{0pt}
    \setlength{\parskip}{6pt plus 2pt minus 1pt}}
}{% if KOMA class
  \KOMAoptions{parskip=half}}
\makeatother
\usepackage{xcolor}
\setlength{\emergencystretch}{3em} % prevent overfull lines
\setcounter{secnumdepth}{5}
% Make \paragraph and \subparagraph free-standing
\ifx\paragraph\undefined\else
  \let\oldparagraph\paragraph
  \renewcommand{\paragraph}[1]{\oldparagraph{#1}\mbox{}}
\fi
\ifx\subparagraph\undefined\else
  \let\oldsubparagraph\subparagraph
  \renewcommand{\subparagraph}[1]{\oldsubparagraph{#1}\mbox{}}
\fi


\providecommand{\tightlist}{%
  \setlength{\itemsep}{0pt}\setlength{\parskip}{0pt}}\usepackage{longtable,booktabs,array}
\usepackage{calc} % for calculating minipage widths
% Correct order of tables after \paragraph or \subparagraph
\usepackage{etoolbox}
\makeatletter
\patchcmd\longtable{\par}{\if@noskipsec\mbox{}\fi\par}{}{}
\makeatother
% Allow footnotes in longtable head/foot
\IfFileExists{footnotehyper.sty}{\usepackage{footnotehyper}}{\usepackage{footnote}}
\makesavenoteenv{longtable}
\usepackage{graphicx}
\makeatletter
\def\maxwidth{\ifdim\Gin@nat@width>\linewidth\linewidth\else\Gin@nat@width\fi}
\def\maxheight{\ifdim\Gin@nat@height>\textheight\textheight\else\Gin@nat@height\fi}
\makeatother
% Scale images if necessary, so that they will not overflow the page
% margins by default, and it is still possible to overwrite the defaults
% using explicit options in \includegraphics[width, height, ...]{}
\setkeys{Gin}{width=\maxwidth,height=\maxheight,keepaspectratio}
% Set default figure placement to htbp
\makeatletter
\def\fps@figure{htbp}
\makeatother

\makeatletter
\makeatother
\makeatletter
\makeatother
\makeatletter
\@ifpackageloaded{caption}{}{\usepackage{caption}}
\AtBeginDocument{%
\ifdefined\contentsname
  \renewcommand*\contentsname{Table of contents}
\else
  \newcommand\contentsname{Table of contents}
\fi
\ifdefined\listfigurename
  \renewcommand*\listfigurename{List of Figures}
\else
  \newcommand\listfigurename{List of Figures}
\fi
\ifdefined\listtablename
  \renewcommand*\listtablename{List of Tables}
\else
  \newcommand\listtablename{List of Tables}
\fi
\ifdefined\figurename
  \renewcommand*\figurename{Figure}
\else
  \newcommand\figurename{Figure}
\fi
\ifdefined\tablename
  \renewcommand*\tablename{Table}
\else
  \newcommand\tablename{Table}
\fi
}
\@ifpackageloaded{float}{}{\usepackage{float}}
\floatstyle{ruled}
\@ifundefined{c@chapter}{\newfloat{codelisting}{h}{lop}}{\newfloat{codelisting}{h}{lop}[chapter]}
\floatname{codelisting}{Listing}
\newcommand*\listoflistings{\listof{codelisting}{List of Listings}}
\makeatother
\makeatletter
\@ifpackageloaded{caption}{}{\usepackage{caption}}
\@ifpackageloaded{subcaption}{}{\usepackage{subcaption}}
\makeatother
\makeatletter
\@ifpackageloaded{tcolorbox}{}{\usepackage[skins,breakable]{tcolorbox}}
\makeatother
\makeatletter
\@ifundefined{shadecolor}{\definecolor{shadecolor}{rgb}{.97, .97, .97}}
\makeatother
\makeatletter
\makeatother
\makeatletter
\makeatother
\journal{Journal Name}
\ifLuaTeX
  \usepackage{selnolig}  % disable illegal ligatures
\fi
\usepackage[]{natbib}
\bibliographystyle{elsarticle-num}
\IfFileExists{bookmark.sty}{\usepackage{bookmark}}{\usepackage{hyperref}}
\IfFileExists{xurl.sty}{\usepackage{xurl}}{} % add URL line breaks if available
\urlstyle{same} % disable monospaced font for URLs
\hypersetup{
  pdftitle={Estructura y funcionamiento de la industria manufacturera de la pesca y la acuicultura nacional.},
  pdfauthor={Andrea Araya Arriagada},
  pdfkeywords={Industria Manufacturera, Concentración},
  colorlinks=true,
  linkcolor={blue},
  filecolor={Maroon},
  citecolor={Blue},
  urlcolor={Blue},
  pdfcreator={LaTeX via pandoc}}

\setlength{\parindent}{6pt}
\begin{document}

\begin{frontmatter}
\title{Estructura y funcionamiento de la industria manufacturera de la
pesca y la acuicultura nacional.}
\author[1]{Andrea Araya Arriagada%
%
}
 \ead{andrea.araya@ifop.cl} 

\affiliation[1]{organization={Instituto de Fomento
Pesquero},addressline={Av. Blanco Encalada
839},city={Valparaíso},postcodesep={}}

\cortext[cor1]{Corresponding author}

        
\begin{abstract}
En la industria manufacturera de la pesca y acuicultura, en 2022
operaron 615 empresas dueñas de 696 plantas, entre Arica hasta
Magallanes. A nivel nacional destacaron las regiones de Los Lagos y
Biobío, siendo la subdivisión de consumo humano la cual concentró la
mayor cantidad de plantas (n = 465), seguida por los Derivados de Algas
(n = 195) y consumo animal (n = 36).
\end{abstract}





\begin{keyword}
    Industria Manufacturera \sep 
    Concentración
\end{keyword}
\end{frontmatter}
    \ifdefined\Shaded\renewenvironment{Shaded}{\begin{tcolorbox}[enhanced, boxrule=0pt, sharp corners, interior hidden, borderline west={3pt}{0pt}{shadecolor}, breakable, frame hidden]}{\end{tcolorbox}}\fi

\hypertarget{introducciuxf3n}{%
\section{Introducción}\label{introducciuxf3n}}

El proyecto Monitoreo Económico de la Industria Pesquera y Acuícola
Nacional tiene como propósito generar información técnica, en el ámbito
económico y social, que la Subsecretaría de Pesca y Acuicultura (SSPA)
requiere para el desarrollo de sus funciones. Este informe reporta los
resultados comprometidos en el segundo objetivo específico del proyecto,
en su versión 2022, que dice relación con la estructura y el
funcionamiento de la industria manufacturera nacional de productos
pesqueros y acuícolas. El análisis se centra en la actividad de
transformación o manufactura, etapa productiva donde confluye la mayor
parte de los desembarques y cosechas. Para caracterizar la estructura de
la industria se utilizaron indicadores de concentración, desigualdad,
inestabilidad e integración, mientras que el funcionamiento se describió
en función del número de plantas, ubicación geográfica y volúmenes de
materia prima y producción.

\hypertarget{metodologuxeda}{%
\section{Metodología}\label{metodologuxeda}}

La industria manufacturera fue clasificada en tres subdivisiones de
acuerdo a los tipos de productos elaborados: consumo humano, consumo
animal y derivados de algas, para una adecuada caracterización del
sector. Desde una perspectiva general, el funcionamiento de cada
subdivisión industrial fue descrita en términos del número de plantas y
su ubicación geográfica, la composición de la materia prima y las líneas
de elaboración, junto al periodo de funcionamiento (días operativos). En
términos de estructura, se utilizaron los elementos clásicos de
caracterización estructural de la industria, concentración e
integración.

Para identificar las empresas que participaron de la actividad
manufacturera, se utilizó el RUT de cada unidad productiva. En este
contexto, se dieron tres opciones: 1) una planta es una unidad
empresarial; 2) si dos o más plantas figuran bajo el mismo RUT, se
acumularon los volúmenes en una única unidad empresarial; y 3) si dos o
más plantas funcionan bajo la figura del multi RUT, se identificó el
Holding que los agrupa, el cual fue considerado como una unidad
empresarial. En la Table~\ref{tbl-a} se presenta un detalle de la
clasificación de las empresas según su RUT, para el 2022.

\hypertarget{tbl-a}{}
\begin{longtable}[]{@{}lrr@{}}
\caption{\label{tbl-a}Número de plantas y empresas, según clasificación
del RUT. 2022.}\tabularnewline
\toprule\noalign{}
tipo & Nro\_plantas & Nro\_empresas \\
\midrule\noalign{}
\endfirsthead
\toprule\noalign{}
tipo & Nro\_plantas & Nro\_empresas \\
\midrule\noalign{}
\endhead
\bottomrule\noalign{}
\endlastfoot
Mismo RUT & 86 & 39 \\
Multi RUT & 45 & 11 \\
RUT individual & 565 & 565 \\
Total & 696 & 615 \\
\end{longtable}

\hypertarget{resultados}{%
\section{Resultados}\label{resultados}}

Durante el 2022, la industria manufacturera del sector pesquero y
acuícola, estuvo conformada por 615 empresas propietarias de 696 plantas
de proceso. Cabe destacar que el 38\% (266) de estas plantas son
unidades de pequeño tamaño productivo (menos de 50 toneladas anuales),
con marcada temporalidad, dedicadas a etapas intermedias de la
elaboración de productos para consumo humano y al secado y/o picado de
algas. La Table~\ref{tbl-b} presenta la distribución del número de
establecimientos por subdivisión industrial y región.

\hypertarget{tbl-b}{}
\begin{longtable}[]{@{}lrrrr@{}}
\caption{\label{tbl-b}Número de establecimientos por subdivisión
industrial y Región. 2022.}\tabularnewline
\toprule\noalign{}
Región & Animal & Derivados de Algas & Humano & Total \\
\midrule\noalign{}
\endfirsthead
\toprule\noalign{}
Región & Animal & Derivados de Algas & Humano & Total \\
\midrule\noalign{}
\endhead
\bottomrule\noalign{}
\endlastfoot
1 & 3 & 19 & 9 & 31 \\
2 & 1 & 52 & 12 & 65 \\
3 & 1 & 39 & 12 & 52 \\
4 & 3 & 33 & 41 & 77 \\
5 & 0 & 8 & 36 & 44 \\
6 & 0 & 5 & 10 & 15 \\
7 & 0 & 2 & 10 & 12 \\
8 & 14 & 15 & 71 & 100 \\
9 & 0 & 0 & 4 & 4 \\
10 & 4 & 6 & 167 & 177 \\
11 & 1 & 1 & 16 & 18 \\
12 & 4 & 3 & 40 & 47 \\
13 & 2 & 5 & 22 & 29 \\
14 & 1 & 1 & 13 & 15 \\
15 & 2 & 1 & 1 & 4 \\
16 & 0 & 5 & 1 & 6 \\
Total & 36 & 195 & 465 & 696 \\
\end{longtable}

A nivel nacional, la manufactura de productos para consumo humano se
llevó a cabo en 465 plantas, donde 41 concentraron el 80\% del volumen,
con un tamaño productivo promedio de 35 mil toneladas al año. Situación
similar se observó en la industria de algas, que reportó actividad
productiva en 195 plantas, 19 de estas concentraron el 50\% del volumen,
con un promedio anual de 3,8 mil toneladas anuales. En el caso de la
industria de harina y aceite, en total 36 plantas reportaron producción,
y 18 dieron cuenta del 91\% del volumen, con un promedio de 34,6 mil
toneladas anuales.

En la Table~\ref{tbl-c} se reporta el tamaño del parque industrial de
manufacturas pesqueras y acuícolas, en el periodo 2018-2022. Respecto
del 2021, se observó un aumento del 10\% en el número de
establecimientos de consumo humano (42 plantas), y una reducción del 7\%
en consumo animal (3 plantas) y del 15\% en derivados de algas (15
plantas).

\hypertarget{tbl-c}{}
\begin{longtable}[]{@{}lrrrrr@{}}
\caption{\label{tbl-c}Número de establecimientos por subdivisión
industrial. 2018-2022.}\tabularnewline
\toprule\noalign{}
Subdivisión & 2018 & 2019 & 2020 & 2021 & 2022 \\
\midrule\noalign{}
\endfirsthead
\toprule\noalign{}
Subdivisión & 2018 & 2019 & 2020 & 2021 & 2022 \\
\midrule\noalign{}
\endhead
\bottomrule\noalign{}
\endlastfoot
Humano & 451 & 484 & 470 & 423 & 465 \\
Animal & 45 & 41 & 39 & 39 & 36 \\
Derivados de Algas & 247 & 238 & 221 & 210 & 195 \\
Total & 743 & 763 & 730 & 672 & 696 \\
\end{longtable}

Los registros referidos al funcionamiento de las instalaciones, muestran
que en el periodo 2018-2022, las plantas permanecieron sin operación
productiva entre un 43\% y 48\% del año, lo cual equivale entre 5 y 6
meses, aproximadamente. El 2022 se observó una contracción en el
promedio de días paralizados, respecto de lo observado entre 2017 y 2019
(Table~\ref{tbl-d}). Según los resultados de la encuesta de operación
industrial (EOI), que recoge datos de la manufactura de la pesca, la
principal causa de paralización de las plantas fue la falta de
abastecimiento.

\hypertarget{tbl-d}{}
\begin{longtable}[]{@{}lrrrrr@{}}
\caption{\label{tbl-d}Cantidad de días paralizados en las plantas
manufactureras. 2018-2022.}\tabularnewline
\toprule\noalign{}
Categoría & 2018 & 2019 & 2020 & 2021 & 2022 \\
\midrule\noalign{}
\endfirsthead
\toprule\noalign{}
Categoría & 2018 & 2019 & 2020 & 2021 & 2022 \\
\midrule\noalign{}
\endhead
\bottomrule\noalign{}
\endlastfoot
Días paralizados & 156 & 170 & 174 & 146 & 148 \\
\end{longtable}


\renewcommand\refname{References}
  \bibliography{bibliography.bib}


\end{document}
